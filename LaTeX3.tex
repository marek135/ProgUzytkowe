\documentclass[a4paper,12pt]{article}

\usepackage[utf8]{inputenc}
\usepackage{amsmath}
\usepackage[MeX]{polski}

%opening
\title{}
\author{}

\begin{document}

\maketitle Programy użytkowe - ćwiczenia 2

\begin{abstract}
\section{Formuły matematyczne w TeXu}
Przetrenuj używanie w TeXu matematycznych formuł i symboli z rozdziału 1 po
czym wykonaj polecenie z rozdziały 2
\subsection{Zapis Matematyczny}
\subsection{Tryb Matematyczny}
Ułamek w tekście $ \frac{1}{x} $ \\
Oto równanie $c^{2}=a^{2}+b^{2}$
%wyeksponowany ułamek i równanie
Ułamek $$ \frac{1}{x} $$ \\
Oto równanie $$c^{2}=a^{2}+b^{2}$$
Tryb matematyczny z użyciem struktury 'equation'
Ułamek
\begin{equation}
\frac{1}{x}
\label{eq:rownanie1}
\end{equation}

Oto równanie

\begin{equation}
c^{2}=a^{2}+b^{2}
\label{eq:rownanie2}
\end{equation}
Można odnieść się do powyższych wzorów wykorzystująć polecenie ’eqref{etykieta}’.
Ułamek ma numer (1) a równanie ma numer (2)
Wiele wzorów w ramach jednego środowiska matematycznego, przy pomocy znaku
’and’ możemy dokonać wyrównania równań 
\begin{align}
\label{eq:partialLW}
\end{align}
\end{abstract}

\section{}

\end{document}